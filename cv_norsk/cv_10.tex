%%%%%%%%%%%%%%%%%%%%%%%%%%%%%%%%%%%%%%%%%
% Friggeri Resume/CV
% XeLaTeX Template
% Version 1.2 (3/5/15)
%
% This template has been downloaded from:
% http://www.LaTeXTemplates.com
%
% Original author:
% Adrien Friggeri (adrien@friggeri.net)
% https://github.com/afriggeri/CV
%
% License:
% CC BY-NC-SA 3.0 (http://creativecommons.org/licenses/by-nc-sa/3.0/)
%
% Important notes:
% This template needs to be compiled with XeLaTeX and the bibliography, if used,
% needs to be compiled with biber rather than bibtex.
%
%%%%%%%%%%%%%%%%%%%%%%%%%%%%%%%%%%%%%%%%%

\documentclass[]{friggeri-cv} % Add 'print' as an option into the square bracket to remove colors from this template for printing

\addbibresource{bibliography.bib} % Specify the bibliography file to include publications

\begin{document}

\header{Andreas}{Kvistad}{Seniorkonsulent} % Your name and current job title/field

%----------------------------------------------------------------------------------------
%	SIDEBAR SECTION
%----------------------------------------------------------------------------------------

\begin{aside} % In the aside, each new line forces a line break
\section{Personlig}
Andreas Henriksen Kvistad
23. September 1993
Fjellhamar
\section{Kontakt}
Nedre Gate 8a, Leilighet 408
0551 Oslo
~
+47 975 46 336
~
\href{mailto:ahk9339@gmail.com}{ahk9339@gmail.com}
\href{https://no.linkedin.com/in/andreas-kvistad-5b7ba6127}{LinkedIn}
\section{Språk}
Norsk
Engelsk Flytende
Tysk Grunnleggende
\section{Programmering}
Python, TypeScript
Java, Spring
React, Vue, HTML, CSS
Azure
\end{aside}


%----------------------------------------------------------------------------------------
%	MYSELF SECTION
%----------------------------------------------------------------------------------------

\section{Kort om meg}
Sivilingeniør i datateknologi med 4 års erfaring som IT-konsulent, med hovedfokus på fullstack utvikling av web-applikasjoner. Jeg er en rolig og hyggelig person som lærer raskt, og er svært interessert i teknologi og programmering. Det faller meg naturlig å ta ansvar både på jobb og utenfor arbeidstid, jeg liker å hjelpe andre, og prøver meg stadig på nye utfordringer. 


%----------------------------------------------------------------------------------------
%	EDUCATION SECTION
%----------------------------------------------------------------------------------------

\section{Utdanning}

\begin{entrylist}

%------------------------------------------------

\entry
{2013--2018}
{Mastergrad {\normalfont Datateknologi, Sivilingeniør}}
{Norges Teknisk-Naturvitenskapelige Universitet, NTNU}
{\emph{Spesialisering innen kunstig intelligens} \\ Femårig sivilingeniørgrad innen datateknologi, hvor jeg valgte å spesialisere meg innen kunstig intelligens. Masteroppgaven handlet om dyp læring med fokus på "one-shot"-læring, mer spesifikt "reinforcement learning" for hurtig bildegjenkjenning.}

%------------------------------------------------

\entry
{2016--2017}
{Utvekslingsår}
{University of California, Santa Barbara}
{Jeg var ett år i California, Santa Barbara, på det fjerde året av utdanningen min. Oppholdet involverte mye prosjektarbeid og innleveringer, og stilte høye krav til å ta ansvar for at fagplan ble fulgt i henhold til NTNU's egen plan.}

%------------------------------------------------

\end{entrylist}

%----------------------------------------------------------------------------------------
%	WORK EXPERIENCE SECTION
%----------------------------------------------------------------------------------------

\section{Erfaring}


%------------------------------------------------

\subsection{Fulltidsansettelser}

\begin{entrylist}

\entry
{2018--}
{Itera}
{Oslo, Nydalen}
{\emph{Seniorkonsulent} \\
Ansatt under stillingstittel "technology analyst" og siden blitt forfremmet til først konsulent, og deretter seniorkonsulent etter 3 års arbeidserfaring. Utleid til KLP siden oppstart høsten 2018, hvor jeg har vært involvert i flere ulike interne prosjekter. Teknologistack har vært React med TypeScript i "frontend", med Java og Spring i "backend". Har etter hvert fått delansvar for applikasjonsarkitektur internt hos KLP, som ett av flere prosjekter. Har også vært en bidragsyter under CMS-migrering, og nå for øyeblikket migrering til sky (Azure).
\\
\\
Tech Lead på sommerprosjekt sommeren 2019 (Gjensidige), 2020 (KLP) og 2021 (KLP). Delt fagansvar for sommerstudenter sommeren 2021, og for nyansatte høsten 2021 og 2022. Holdt tekniske fagintervjuer for diverse utviklerkandidater opp til seniorkonsulentnivå.
}

%------------------------------------------------

\entry
{2012 -- 2013}
{Benterud Skole (Barneskole)}
{Benterud Skole, Lørenskog}
{\emph{Læringsassistent, SFO-ansatt, Vikar} \\
Jobbet på Benterud skole hovedsaklig som læringsassistent for barn med lærevansker, eller som en ressurs for hovedlæreren. Vikarierte noe i diverse fag, og jobbet mye på SFO for elever opp til fjerde skoletrinn.}


\end{entrylist}

\subsection{Sommeransettelser}

\begin{entrylist}

\entry
{2016 \& 2017}
{Statnett SF}
{Oslo, Nydalen}
{\emph{Sommeransatt/Ekstrahjelp} \\
Todelt jobb i et lag bestående av to personer. Den ene delen bestod av å måle jordingen på høyspentmaster, med andre ord deres evne til å lede potensielt farlig strøm vekk fra mast og trygt ned i bakken. Den andre delen bestod av å planlegge og tegne jording på master som skal bygges senere. Dette innebærte å måle jordens relative jordresistivitet, og på bakgrunn av målingene avgjøre hvor mye jording som skal bli lagt ved masten. Dette ble så tegnet inn på en skisse som dekket området rundt masten. Mesteparten av arbeidet ble gjort i felt, ved diverse høyspentlinjer rundt om i landet.
\\
\\
Ansvarlig for opplæring av et annet lag. Opplæringen varte i \'{e}n uke, hvor jeg overså arbeidet de gjorde, og hjalp til når det var nødvendig.}

\end{entrylist}

%------------------------------------------------



\subsection{Delttidsansettelser}

\begin{entrylist}

%------------------------------------------------

\entry
{2010 -- 2012}
{Losby Gods}
{Losby Gods, Lørenskog}
{\emph{Servicemedarbeider} \\
Jobbet som servitør/servicemedarbeider i store selskaper som bryllup, sammen med andre servitører, eller som ansvarlig for egne bord (\'{a} la carte) i restauranten. Arbeidet involverte mye samhandling med kunder og andre ansatte.}

%------------------------------------------------


\end{entrylist}



%------------------------------------------------


%----------------------------------------------------------------------------------------
%	CERTIFICATIONS SECTION
%----------------------------------------------------------------------------------------

\section{Sertifiseringer}
\begin{entrylist}
\entry
{AZ-900}
{\href{https://docs.microsoft.com/en-us/certifications/exams/az-900}{Azure Fundamentals}}
{Oslo}
{Sertifiseringsbevis kan sendes etter ønske.}

\end{entrylist}


%----------------------------------------------------------------------------------------
%	PUBLICATIONS SECTION
%----------------------------------------------------------------------------------------

\section{Publiseringer}
\begin{entrylist}
\entry
{Mastergrad}
{\href{https://arxiv.org/abs/1909.01757}{Augmented Memory Networks for Streaming-Based Active One-Shot Learning}}
{Oslo}
{Resulterende publisering av etterarbeid på mastergrad i 2018.}

\end{entrylist}

%----------------------------------------------------------------------------------------
%	REFERENCES SECTION
%----------------------------------------------------------------------------------------

\section{Referanser}
Oppgis på forespørsel.
%\begin{entrylist}
%
%%----------------------------------------------------------------------------------------
%
%\entry
%{Itera}
%{Øyvind Berntsen Kheradmandi}
%{Oslo, Nydalen}
%{\emph{Referanseperson}, Øyvind Berntsen Kheradmandi, Team Lead
%\\
%Kontakt:
%\begin{itemize}
%\item +47 909 95 149
%\item arild.berstad@statnett.no
%\end{itemize}
%}
%
%%----------------------------------------------------------------------------------------
%
%\entry
%{Statnett SF}
%{Arild Kvamme Berstad}
%{Oslo, Nydalen}
%{\emph{Attest} skrevet av Arild Kvamme Berstad, Ingeniør ved UTLE - Statnett SF, kan sendes etter ønske.
%\\
%Kontakt:
%\begin{itemize}
%\item +47 909 95 149
%\item arild.berstad@statnett.no
%\end{itemize}
%}
%
%
%%----------------------------------------------------------------------------------------
%
%\entry
%{Benterud}
%{Lillian Lauritzen}
%{Benterud Skole, Lørenskog}
%{\emph{Referanseperson}, Lillian Lauritzen, Vikarinnkaller
%\\
%Kontakt:
%\begin{itemize}
%\item +47 977 90 764
%\item lillau@lorenskog.kommune.no
%\end{itemize}
%}
%
%%----------------------------------------------------------------------------------------
%
%\entry
%{Losby Gods}
%{Heidi Fjellheim}
%{Losby Gods, Lørenskog}
%{\emph{Referanserperson}, Heidi Fjellheim, Direktør
%\\
%Kontakt:
%\begin{itemize}
%\item +47 970 69 620
%\item h.fjellheim@losbygods.no
%\end{itemize}
%}
%
%\end{entrylist}

%----------------------------------------------------------------------------------------

\end{document}